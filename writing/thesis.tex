\documentclass[12pt,leqno,a4paper]{article}
% Minimalbeispiel für Diplom- oder Studienarbeit in Latex
% OHNE GEWAEHR
% Antje Schweitzer, Juli 2011 & Juli 2014 
% folgende Dateien gehoeren zu diesem Beispiel:
% - thesis.tex (der eigentliche Text)
% - references.bib (die "Datenbank" mit den Literaturverweisen im Bibtex-Format)
% - example.pdf (eine Beispielgrafik)
% damit kann man dann nach der Anleitung unten folgendes Dokument erzeugen:
% - thesis.pdf 

% Minimal example for student papers in LaTeX, can be used as a template
% without warranty
% (English "translation" by Nils Reiter, July 2014)
% You need multiple files for this example to work:
% - thesis.tex (this file)
% - references.bib (a BibTeX file containing the bibliographic entries)
% - example.pdf (an example figure)

% ANLEITUNG -- INSTRUCTIONS
% To create a PDF, please run the following steps:
% - pdflatex thesis.tex
% - bibtex thesis
% - pdflatex thesis.tex
% - pdflatex thesis.tex
% (yes, some steps need to be run multiple times, for reasons)

% Antje Schweitzer, Oct. 2016 - updated statement of authorship
% 
% Antje Schweitzer, Dec 2020 
% - updated translation of statement of authorship ;)
% - made title fit into the window prescribed for Computer Science theses
% please note that for the Computer Science students, the statements
% have slightly different wordings and need to be on the last page
% please check requirements by C.S. department in that case
% (also no guarantee that the window is correct -
% the version I compiled and printed does fit)
% - changed to UTF-8 encoded .tex file, included inputenc utf8, 
% so Umlauts can be typed as usual

\usepackage{natbib}
\usepackage{epsfig}
\usepackage{booktabs}
\usepackage[paper=a4paper,left=3cm,right=3cm]{geometry}% http://ctan.org/pkg/geometry
\usepackage[utf8]{inputenc} 
\usepackage[english,german]{babel}
\usepackage{dingbat}
\usepackage{amsmath}
\usepackage{amsfonts}
\usepackage{float}
\usepackage{tipa}

\renewcommand{\baselinestretch}{1.3}
\parskip = \medskipamount
\frenchspacing
\bibpunct[; ]{(}{)}{;}{a}{,}{;}


\newcommand{\Titel}{Phonetic Representations of Speech\\ 
for Human Pronunciation Feedback and Automatic Accent Transfer}


\begin{document}

% larger left margin for title page to get it centered into C.S. template

\begin{titlepage}
  % different margin so titlepage fits into template
  \newgeometry{left=3.5cm,right=2cm}
  \large
   \begin{center}
    Institut für Maschinelle Sprachverarbeitung\\
    Universität Stuttgart\\
    Pfaffenwaldring 5B\\
    D-70569 Stuttgart\\    
    
    \vspace{2.5cm}
    Master thesis\\
   % \vfill
    {\LARGE \bf \Titel} \\
    \vspace{2cm}
    Isaac Riley\\
       \vfill
    \begin{tabular}[t]{lr}
    Studiengang: & M.Sc. Computational Linguistics \\ % oder: B.Sc. Maschinelle Sprachverarbeitung, M.Sc. Informatik, ...\\
    \\
    \\
    {Prüfer*innen:} & Prof. Dr. Wolfgang Wokurek\\
     & Prof. Dr. Antje Schweitzer\\
    {Betreuer:} & Prof. Dr. Wolfgang Wokurek\\ 
    \\
    \\
    {Beginn der Arbeit:} & 01.04.2021\\
    {Ende der Arbeit:} & 01.10.2021\\
    \end{tabular}
  \end{center}
\setlength{\hoffset}{0cm}

  \normalsize
\end{titlepage}

% back to original geometry
\newgeometry{left=3cm,right=3cm}

\newpage
\thispagestyle{empty}


% note that for Computer Science theses (not Maschinelle Sprachverarbeitung or Computational Linguistics)
% this statement has a different wording and needs to be on the last rather than the first page. 
% Please adapt accordingly. 
\begin{otherlanguage}
{german}
\noindent\textbf{Erklärung (Statement of Authorship)}\\


\noindent Hiermit erkläre ich, dass ich die vorliegende Arbeit selbstständig verfasst habe und dabei keine andere als die angegebene Literatur verwendet habe. Alle Zitate und sinngemäßen Entlehnungen sind als solche unter genauer Angabe der Quelle gekennzeichnet. Die eingereichte Arbeit ist weder vollständig noch in wesentlichen Teilen Gegenstand eines anderen Prüfungsverfahrens gewesen. Sie ist weder vollständig noch in Teilen bereits veröffentlicht. Die beigefügte elektronische Version stimmt mit dem Druckexemplar überein.%% delete the following for theses in German; 
%% delete or keep for English theses. 
%% The German version above must be retained 
%% and signed even if the thesis is written
%% in another language than German.
\end{otherlanguage}
\footnote{Non-binding translation for convenience: This thesis is the result of my own independent work, and any material from work of others which is used either verbatim or indirectly in the text is credited to the author including details about the exact source in the text. This work has not been part of any other previous examination, neither completely nor in parts. It has neither completely nor partially been published before. The submitted electronic version is identical to this print version.}\\[2cm]
\vspace{2cm}
(Isaac Riley)

\newpage
\thispagestyle{empty}
\noindent \textbf{Acknowledgments}\\
\noindent XXX

\newpage
\tableofcontents
\newpage

\section{Introduction}
In linguistics, the notion of accent refers to a pattern of pronunciation that does not change 
the semantic content of what is uttered, but which may carry pragmatic meaning and 
convey demographic information about the speaker. While some phonetic variation may be random 
or occur at the individual level (idiolect), accent typically varies systematically according 
to native language and dialect, which in turn vary geographically and demographically.

Accent is not usually an impediment to human understanding of speech; this is especially true 
among native speakers. In automatic speech recognition, non-standard accent typically has a much 
stronger negative effect on recognition accuracy. For this reason, accent is an active subject of 
research in automatic speech processing.
%accent transfer and reduction
%accent recognition
%voice conversion and voice cloning

Related to accent transfer is the task of voice conversion. The goal of voice conversion is to 
generate speech containing identical linguistic information as an input sample, modifying only the 
timbre to match a target speaker.

\section{Background}
\subsection{Automatic Speech Recognition}
Automatic speech recognition 
\subsection{Text-to-Speech Synthesis}

\subsection{Speech Signal Processing}
\subsubsection{Short-Time Fourier Transform}
%https://www.youtube.com/watch?v=UKHBWzoOKsY

\subsubsection{Mel Spectrum and Coefficients}
%https://en.wikipedia.org/wiki/Mel-frequency_cepstrum
%https://en.wikipedia.org/wiki/Mel_scale
%http://practicalcryptography.com/miscellaneous/machine-learning/guide-mel-frequency-cepstral-coefficients-mfccs/
%https://medium.com/analytics-vidhya/understanding-the-mel-spectrogram-fca2afa2ce53
%https://towardsdatascience.com/getting-to-know-the-mel-spectrogram-31bca3e2d9d0
%https://de.wikipedia.org/wiki/Mel_Frequency_Cepstral_Coefficients
\subsubsection{}
\subsubsection{}

\subsection{Neural Networks}
\subsubsection{Feedforward Neural Networks}
A feedforward neural network 

\subsubsection{Recurrent Neural Networks}
Feedforward neural networks are effective in processing vector inputs. If the input 
data

One powerful and popular architecture for recurrent neural networks is the long
short-term memory (LSTM) network.

\subsubsection{Convolutional Neural Networks}
When input data have a spatial component, it is beneficial to use a network architecture 
that 

\subsubsection{Generative Adversarial Networks}
Operating within the parradigm of supervised learning, neural networks
can be used for classification by learning a posterior distribution over 
labels, given numerical features representing features of an input sample.
%The model is trained by backpropagation, in which errors in classification 
%are appropriately attributed to each of its parameters.
Given sufficient exposure to labeled training data, an appropriately designed 
model can update its parameters so as to assign an increasingly larger probability
to the correct label.

Another branch of machine learning involves generative learning. Given a training dataset,
a generative model learns to output samples resembling those from the training dataset. 

In ``vanilla'' generative adversarial networks (GANs), the are two networks whose training 
takes the form of a minimax game, in which each model seeks to maximize the loss of the other.
The first network is the generator, which is given random noise as input and tasked with 
outputting samples of the same type as the target data. The second model is the discriminator,
a classification network tasked with classifying generated samples as real or generated. 
Throughout the (ideal) adversarial training process, the generator 

\section{Related Work}
\subsection{Models Used}
\subsubsection{Tacotron 2}
Tacotron 2 is a deep neural sequence-to-sequence model that generates mel spectrograms from text. 
I can be combined with a deep neural vocoder, it forms an end-to-end text-to-speech system.
\subsubsection{WaveGlow}
WaveGlow is a deep neural flow-based vocoder that generates waveform audio from mel spectrograms.
\subsubsection{Allosaurus}

\subsubsection{CycleGAN}
In `vanilla' GANs, the input is typically random noise, which is mapped to some point in the target distribution. 
CycleGAN

\subsubsection{SeqGAN}



\section{Resources}

\newpage
\section{Methods}

\subsection{Phonetic Representation-Based Speech Synthesis and Accent Transfer}
%To learn representations of input speech samples, I use a neural network, 
or rather a family of similar neural networks all sharing the same 
sequence-to-sequence architecture. This architecture consists of an 
encoder, a decoder, and an attention mechanism, each of which will be discussed 
in greater detail below.

%%%%%%%%%%%%%%%%%%%%%%%%%%%%%%%%%%%%%%%%%%%%%%%%%%%%%%%%%%%%%%%%%%%%%%%%
\subsubsection{Phone Recognition Model}
\paragraph{Encoder} %---------------------------------------------------
The audio files in .wav format are pre-processed into 80-dimensional filter
banks to be input into the encoder. Additionally, in some experiments between 
2 and 4 frames are ``stacked''. Thus, for an input sample with $N$ frames, 
stacking would result in and $N/2$ frames each with dimensionality of 160; 
stacking 3 frames results in $N/3$ frames each with dimensionality 240, and 
so on.

The encoder is a 


\paragraph{Decoder} %---------------------------------------------------
The initial decoder state, like the initial encoder state, contains learnable 
parameters; each successive state is derived from the previous decoder state and 
from the encoder output.

At time step $t$, the decoder's task is to predict the next output, $\hat{y}_t$.
To do this it uses the embedding of the last prediction, $m_t = W^E(\hat{y}_{t-1})$.

$$ \textbf{h}_t = \textrm{GRU} \left( \begin{bmatrix}
    \textbf{m}_{t-1} \\ \textbf{o}_{t-1}
\end{bmatrix}, \textbf{h}_{t-1} \right)$$

$$ \textbf{o}_t =  \textrm{ReLU} \left( W^o \begin{bmatrix}
    \textbf{a}_t \\ \textbf{h}_t
\end{bmatrix} \right) $$

$$ \hat{\textbf{y}}_t = W^C \textbf{o}_t $$

$$  $$

\paragraph{Attention Mechanism} %---------------------------------------
The attention mechanism in deep learning is a fairly recent innovation [XXX]
enabling the neural network to, for each unit of output, distinguish 
certains parts of the input as more relevant to that output, thus emulating the 
human tendency to focus on portions of the input, rather than dividing one's attention
uniformly over the entire input (as earlier encoder-decoder arcitectures were required 
to do by default).

Attention can be thought of as a kind of query system, in which the last predicted
unit of output can be used to `query' the output of the encoder and create a weighted
sum of the encoder (source) states for further use in the decoder.

Formally, at time step $t$ in the decoding process,
$$ \alpha_t = \textrm{softmax}\left( \textrm{ReLU}( W^e S^e + W^h \textbf{h}^t [1]_{1 \times N} ) \textbf{v} \right). $$
where $S^e$ is the matrix of encoder state vectors, 
$\textbf{s}^d$ is the current decoder state vector, the $W$ are the corresponding 
weight matrices, and $\textbf{v}$ is a weight vector. 
The row vector $[1]_{1 \times N}$ serves simply to `copy' the single target state vector 
to be added to each of the encoder state vectors.
We compute the attention values as a weighted sum of source states:
$$ \textbf{a}_t = \sum\limits_{i=1}^{N} \alpha_{ti} \textbf{s}_i^{\tiny \textrm{source}}. $$
Note that $t$ is used to index decoder steps, while $i$ indexes to encoder steps.
%A weight vector $\alpha$ is calculated 

For convenience, the model notation used is summarized below:
\begin{itemize}
    \item $X $: input consisting of $N$ frames of $80k$-dimensional FBANK features, where $k$ is the parameter indicating how many consicutive frames are to be stacked \\
    \item $Y $: $V \times M $ output matrix consisting of $M$ $V$-dimensional one-hot vectors corresponding to phone IDs; $V$ is vocabulary size \\
    \item $W^e \in \mathbb{R}^{h \times 2h}$: attention weight matrix for encoder state matrix
    \item $W^h \in \mathbb{R}^{h \times h}$: attention weight matrix for target state vector 
    \item $\textbf{h}_t $: decoder hidden state at time $t$
    \item $t $: decoder time step; corresponds to output phone number
    \item $i $: encoder time step; corresponds to input frames 
    \item $\hat{\textbf{y}}_t$: prediction for phone $t$
    \item $\textbf{m}_t $: embedding of prediction $\hat{y}_t$ or ground truth $y_t$
    \item $W^m$: embedding matrix used to compute $\textbf{m}_t$
    \item $W^c $: classifier weights, used to generate a prediction from the output vector
    \item $a_t $: attention vector at time step $t$ 
    \item $S$: matrix of encoder state targets 
    \item $ $ 
    \item $ $ 
    \item $ $ 
    \item $ $ 
    \item $ $ 
    \item $ $ 
    \item $ $ 
    
\end{itemize}



%%%%%%%%%%%%%%%%%%%%%%%%%%%%%%%%%%%%%%%%%%%%%%%%%%%%%%%%%%%%%%%%%%%%%%%%
\subsubsection{Articulatory Feature Models}

Both the binary and contintuous articulatory feature models, (henceforth
simply the binary and continuous models for brevity), retain the basic 
structure and approach of the phone recognition model. The primary difference
is that rather than predicting a discrete sequence of phone, these models 
predict articulatory features rather than phones. This provides greater 
universality because the articulatory features allow for the description
of any realizable phone\footnote{Some extremely rare articulatory features 
have been omitted due to a lack of suitable data; however the phone set used here is capable 
of fully describing all high-resource and nearly all medium-resource languages.}, 
and the number of features is less than the phonetic inventory of nearly 
all langages, which allows for more compact representations.

The articulatory features are summarized in Table \ref{table:artfeats}.
\begin{table}
\begin{center}
\begin{tabular}{|l|l|c||l|l|c|} \hline
    Group & Binary Feature & Continuous? & Group & Binary Feature & Continuous? \\ \hline
    Quality & Voiced & \checkmark & Place & Front &  \checkmark \\
    Class & Silence & \checkmark & & Near-Front & \\
    & Vowel & \checkmark & & Central  &  \\
    & Pulmonic & \checkmark & & Near-Back &  \\ 
    & Pulmonic Consonant & \checkmark & & Back & \\
    & Non-pulmonic Consonant & \checkmark & & Close &  \\
    Style & Rounded & \checkmark & & Near-Close &  \\
    & Rhoticized & \checkmark & & Close-Mid &  \\
    Manner & Plosive & \checkmark & & Open-Mid &  \\
    & Nasal & \checkmark & & Near-Open &  \\
    & Trill & \checkmark & & Open &  \checkmark \\
    & Tap/Flap & \checkmark & & Schwa &  \\
    & Fricative & \checkmark & & Bilabial &  \checkmark \\
    & Lateral Fricative & \checkmark & & Labiodental &  \checkmark \\
    & Affricate & \checkmark & & Dental &  \checkmark \\
    & Approximant & \checkmark & & Alveolar &  \checkmark \\
    & Lateral Approximant & \checkmark & & Postalveolar &  \checkmark \\
    & Click & \checkmark & & Retroflex &  \checkmark \\
    & Implosive & \checkmark & & Palatal &  \checkmark \\
    & Ejective & \checkmark & & Velar &  \checkmark \\
    & & & & Uvular &  \checkmark \\
    & & & & Pharyngeal &  \checkmark \\
    & & & & Glottal &  \checkmark \\
    & & & & Lateral &  \checkmark \\ \hline 
\end{tabular}
\end{center}
\caption{Binary and Continuous Articulatory Features}
\label{table:artfeats}
\end{table}

\paragraph{Encoder} %---------------------------------------------------


\paragraph{Decoder} %---------------------------------------------------


\paragraph{Attention Mechanism} %---------------------------------------





\input{acoustic_model.tex}

\subsubsection{Speech Synthesis from Phonetic Representations}
\input{vocoder_model_architecture}

\subsubsection{Accent Transfer in Phonetic Representation Space}
\input{gan_model_architecture}

\subsection{Phonetic Transcription-Based Speech Synthesis and Accent Transfer}
\subsubsection{Transcription}


\subsubsection{Rule-Based Segmental Accent Transfer}
Because accents tend to be characterized by a number of salient features differing
from the `standard' dialect or simply from other dialects, it is possible to formulate
mapping rules that define which phonetic changes must be made to a source accent to make it sound
more like a target accent. For example, a speaker of Received Pronunciation English wishing 
to imitate a speaker of American English will typically pronounced syllable-final ``\textit{r}'' 
as [\textipa{\*r}] and will voice intervocalic occurrences of ``\textit{t}''. An American English 
speaker wishing to imitate RP English might apply the inverse of these rules. 

The following section seeks to investigate and formalize approaches to learning 
phonetic transformation rules from non-parallel data.

\subsubsection{GAN-Based Segmental Accent Transfer}

\section{Results}


\section{Conclusions}


%%%%%%%%%%%%%
% Bibliographie
\bibliographystyle{plainnat}
\bibliography{references}



\end{document}
 
