In linguistics, the notion of accent refers to a pattern of pronunciation that does not change 
the semantic content of what is uttered, but which may carry pragmatic meaning and 
convey demographic information about the speaker. While some phonetic variation may be random 
or occur at the individual level (idiolect), accent typically varies systematically according 
to native language and dialect, which in turn vary geographically and demographically.

Accent is occasionally an impediment to human understanding of speech; however, among native 
and advanced speakers, accent is often no impediment to communication.
In automatic speech recognition, accent accents underrepresented in training datasets 
typically have a much stronger negative effect on recognition accuracy. For this reason, 
accent is an active subject of research in automatic speech processing.
%accent transfer and reduction
%accent recognition
%voice conversion and voice cloning

Related to accent transfer is the task of voice conversion. The goal of voice conversion is to 
generate speech containing identical linguistic information as an input sample, modifying only the 
timbre to match a target speaker. In some respects, voice conversion is a simpler task; namely, 
there is no temporal realignment and phonetic information remains the same. 
The absence of temporal realignment means that voice conversion models have an equal number of frames 
in input and output. This makes possible frame-to-frame mappings, also surrounding context must still 
be taken into account in order to achieve good results.

This works explores approaches to accent conversion with and without voice conversion. 
These approaches may be referred respectively as intra-speaker accent conversion and cross-speaker 
accent conversion.
In intra-speaker accent conversion (henceforth simply \textit{accent conversion}), an utterance from the source speaker is modified to match the accent 
of the target speaker, without changing voice quality or non-accent-carrying linguistic content. 
By contrast, in cross-speaker accent conversion, accent conversion is combined with voice conversion. Henceforth, this will 
simply be referred to as speaker conversion.

This work investigates two distinct but related approaches to speech conversion, both accent conversion and 
speaker conversion. The first is a continuous approach in which speech is mapped to and synthesized from a
continuous phonetic representation. The second is a discrete approach, in which speech is transcribed to and 
synthesized from the International Phonetic Alphabet (IPA).

Clearly, in the simplest case, speaker conversion may involve the combination of ASR and TTS, where the ASR model 
recognizes the speech of the source speaker and the TTS model synthesizes the speech of the target speaker from that same text.
This work takes an alternative approach, in which speech is transcribed in the IPA, 
modified by an intermediate model to match the phonetic patterns of the target speaker, and a TTS model trained 
to synthesize speech from IPA is used to synthesize the desired speech.

\begin{center}
  \textbf{Summary and Comparison of Speech Conversion Tasks}\\~\\
  \begin{tabular}{l|c|c|c}
    %\caption{Intra-Speaker Accent Conversion}
      Speech Aspect   & Accent Conversion & Voice Conversion & Speaker Conversion \\\hline
      content         & source & source & source \\
      accent          & \textbf{target} & source & \textbf{target} \\
      vocal quality   & source & \textbf{target} & \textbf{target} \\
    \end{tabular}
  \end{center}
