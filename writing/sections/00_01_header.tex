\documentclass[12pt,leqno,a4paper]{article}


\usepackage{natbib}
\usepackage{epsfig}
\usepackage{booktabs}
\usepackage[paper=a4paper,left=3cm,right=3cm]{geometry}% http://ctan.org/pkg/geometry
\usepackage[utf8]{inputenc} 
\usepackage[english,german]{babel}
\usepackage{dingbat}
\usepackage{amsmath}
\usepackage{amsfonts}
\usepackage{float}
\usepackage[extra]{tipa}
\usepackage{dingbat}
\usepackage{algorithm}
\usepackage{algpseudocode}

% Minimalbeispiel für Diplom- oder Studienarbeit in Latex
% OHNE GEWAEHR
% Antje Schweitzer, Juli 2011 & Juli 2014 
% folgende Dateien gehoeren zu diesem Beispiel:
% - thesis.tex (der eigentliche Text)
% - references.bib (die "Datenbank" mit den Literaturverweisen im Bibtex-Format)
% - example.pdf (eine Beispielgrafik)
% damit kann man dann nach der Anleitung unten folgendes Dokument erzeugen:
% - thesis.pdf 

% Minimal example for student papers in LaTeX, can be used as a template
% without warranty
% (English "translation" by Nils Reiter, July 2014)
% You need multiple files for this example to work:
% - thesis.tex (this file)
% - references.bib (a BibTeX file containing the bibliographic entries)
% - example.pdf (an example figure)

% ANLEITUNG -- INSTRUCTIONS
% To create a PDF, please run the following steps:
% - pdflatex thesis.tex
% - bibtex thesis
% - pdflatex thesis.tex
% - pdflatex thesis.tex
% (yes, some steps need to be run multiple times, for reasons)

% Antje Schweitzer, Oct. 2016 - updated statement of authorship
% 
% Antje Schweitzer, Dec 2020 
% - updated translation of statement of authorship ;)
% - made title fit into the window prescribed for Computer Science theses
% please note that for the Computer Science students, the statements
% have slightly different wordings and need to be on the last page
% please check requirements by C.S. department in that case
% (also no guarantee that the window is correct -
% the version I compiled and printed does fit)
% - changed to UTF-8 encoded .tex file, included inputenc utf8, 
% so Umlauts can be typed as usual

\renewcommand{\baselinestretch}{1.3}
\parskip = \medskipamount
\frenchspacing
\bibpunct[; ]{(}{)}{;}{a}{,}{;}


\newcommand{\Titel}{Phonetic Representations of Speech\\ 
for Human Pronunciation Feedback and Automatic Accent Transfer}
