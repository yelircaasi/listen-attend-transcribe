In natural human language, the correspondence between orthography and pronunciation 
is not entirely one-to-one; in some languages, such as English or Tibetan, it is 
difficult to predict pronunciation from spelling and vice versa. In some languages, 
such as Spanish or Swahili, there is a significantly higher degree of predictability, 
but the dialectical variation means that this tight correspondence does not hold for 
all groups of speakers. Additionally, even among languages using the same scripts, 
the same character often represents differents sounds in different languages. However, 
even within a language, characters can represent different speech sounds depending on 
surrounding letters or position within a word.

For all of these reasons, it is important to have an agreed-upon convention for 
representing speech sounds. The de facto standard is the International Phonetic Alphabet 
\citep{ipa,handbookofphonetics}, 
which was developed to provide a language-agnostic (as well as dialect-agnostic) means 
for phonetic transcription.
This makes it an excellent tool for transcribing accents in which differences in pronunciation 
are of interest.

The IPA is conceived to be able to represent, in theory, all phones appearing in any 
known language, or indeed, any phone realizable by humans. In the present work, because 
the language under consideration is English (albeit with two starkly differing accents), 
the full power of the IPA is not required. All segments considered are either vowels 
or consonants, and all consonants are pulmonic, i.e. produced by exhaling air from the lungs.

Vowels are classified along three key dimensions: openness/closeness, backness/frontness,
and roundedness. The latter is binary, while the other two are continuous.
Consonants are classified by place of articulation, manner of articulation, and voicedness. 
For example, \textipa{b} is the voiced bilabial plosive, while \textipa{p} is its unvoiced 
counterpart. Realized at a different place of articulation, \textipa{g} and \textipa{k} 
are the velar plosives (also respectively voiced and unvoiced). 
Alternatively, holding the place of articulation constant and changing the manner of 
articulation to fricative yields the pair \textipa{B, F}. All phonetic segments used in this work fit this schema.