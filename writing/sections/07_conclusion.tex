%Possibility of IPA speech synthesis?
%Possibility of GANs on intermediate represenation?
%Possibility of speech synthesis involving spontaneous speech?
% --> need to annotate pauses
%

It should be noted that this work does not make any normative prescriptions about a ``correct'' dialect of English,
nor does it condone such prescriptions. In the author's view, different dialects are equally valid.
Applications of accent transfer, such as accent reduction, may be motivated by practical concerns, such as the desire to decrease the difficulty 
of understanding or being understood by an interlocutor. However, these considerations do not imply the superiority 
of any accent over another. The focus on one of the two theoretically possible directions of transfer, 
namely Indian English to American English, was motivated by considerations regarding the phones available in each.

This work, unfortunately, was not able to improve upon the results of \cite{facviappg} regarding speech synthesized 
from PPGs. However, this work does suggest a number of directions for future research.  an interesting direction for future work might be to work with multi-accent or even 
multi-language acoustic models. Conceivably, speakers with different accents would be distinguishable in such a phonetic space, rendering the transfer models 
more useful.

The failure of the modified Tacotron models to synthesize coherent speech is also in need of further investigation. 
One recommended line of inquiry, which this work would have explored if time had permitted, regards whether 
prompted speech by a professional voice actor would yield better results.

The two most interesting novel contributions in this work are the demonstration of speech synthesis from 
IPA, and the presentation of algorithms for transcription-to-transcription mappings. 
Each of these also points to the potential for interesting work to be done. Neural speech synthesis from IPA, 
while obviously more challenging than established methods, offers exciting possibilities. 
It would represent a further step toward the ideal of a multilingual, multi-speaker speech synthesis system, which 
in turn would play an important part in end-to-end spech translation.
It also has the potential to be used in smaller practical applications, such as accent transfer and 
human accent training.


*** TO BE ADDED - WAITING FOR RESULTS ***
