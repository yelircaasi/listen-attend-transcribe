%Possibility of IPA speech synthesis?
%Possibility of GANs on intermediate represenation?
%Possibility of speech synthesis involving spontaneous speech?
% --> need to annotate pauses
%

It should be noted that this work does not make any normative prescriptions about a ``correct'' dialect of English,
nor does it condone such prescriptions. In the author's view, different dialects are equally valid.
Applications of accent transfer, such as accent reduction, may be motivated by practical concerns, such as the desire to decrease the difficulty 
of understanding or being understood by an interlocutor. However, these considerations do not imply the superiority 
of any accent over another. The focus on one of the two theoretically possible directions of transfer, 
namely Indian English to American English, was motivated by considerations regarding the phones available in each.

This work, unfortunately, was not able to improve upon the results of XXX regarding speech synthesized from PPGs. However, an interesting direction 
for future work might be to work with multi-accent or even multi-language acoustic models. Conceivably, 
speakers with different accents would be distinguishable in such a phonetic space, rendering the transfer models 
more useful.

The two most interesting novel contributions in this work are the demonstration of speech synthesis from 
IPA, and the presentation of a method for transcription-to-transcription mappings.

*** TO BE ADDED - WAITING FOR RESULTS ***
